\chapter{\label{ch:ch01}ТЕОРЕТИЧЕСКАЯ ЧАСТЬ} % Нужно сделать главу в содержании заглавными буквами

\section{Дендрит}
\textbf{Дендри́ты}~\cite{wikiDen} (от греч. δένδρον — дерево) — сложнокристаллические образования древовидной ветвящейся структуры.
\subsubsection{Формирование} %не нужен секшен
Дендрит представляет собой ветвящееся и расходящееся в стороны образование, возникающее при ускоренной или стеснённой кристаллизации в неравновесных условиях, когда кристалл расщепляется по определённым законам. В результате он утрачивает свою первоначальную целостность, появляются кристаллографически разупорядоченные блоки. Они ветвятся и разрастаются в разные стороны подобно дереву, тянущемуся к солнечному свету, кристаллографическая закономерность изначального кристалла в процессе его дендритного развития утрачивается по мере его роста. Дендриты могут быть трёхмерными объёмными (в открытых пустотах) или плоскими двумерными (если растут в тонких трещинах горных пород).

Процесс образования дендрита принято называть дендритный рост.

Самый крупный дендрит был обнаружен в конце XIX века в 100-тонном слитке стали и был назван в честь русского учёного Дмитрия Константиновича Чернова, детально исследовавшего процесс зарождения и роста кристаллов (в частности, дендритных стальных кристаллов). Вес «кристалла Д. К. Чернова» составил 3,45 кг, длина — 39 см, химический состав — 0,78 \% углерода, 0,255 \% кремния, 1,055 \% марганца, 97,863 \% железа.
\subsubsection{Разновидности}
В качестве примера дендритов можно привести снежинки, ледяные узоры на оконном стекле, живописные окислы марганца, имеющие вид деревьев в пейзажных халцедонах («моховой агат») и в тонких трещинах розового родонита. Другие примеры — веточки самородной меди в зонах окисления рудных месторождений, дендриты самородных серебра и золота, решётчатые дендриты самородного висмута и ряда сульфидов. Почковидные или кораллообразные дендриты известны для малахита, барита и многих других минералов, к ним относятся и так называемые «пещерные цветы» кальцита и арагонита в карстовых пещерах.
\section{FLTK} %обзор
Fast, Light Toolkit~\cite{wikiFL} — кросс-платформенная библиотека инструментов с открытым исходным кодом (лицензия LGPL) для построения графического интерфейса пользователя (GUI). FLTK произносится как «фултик».

Изначально разрабатывалась Биллом Спицтаком (Bill Spiztak). FLTK создавалась для поддержки 3D графики и поэтому имеет встроенный интерфейс к OpenGL, но хорошо подходит и для программирования обычных интерфейсов пользователя.

Библиотека использует свои собственные независимые системы виджетов, графики и событий, что позволяет писать программы одинаково выглядящие и работающие на разных операционных системах. В отличие от других подобных библиотек (Qt, GTK, wxWidgets) FLTK ограничивается только графической функциональностью. Поэтому она имеет малый размер и обычно компонуется статически (это исключение из лицензии GNU Lesser General Public License, разрешенное разработчиками). FLTK не использует сложных макросов, препроцессоров и продвинутых возможностей языка C++ (шаблоны, исключения, пространства имен). Вкупе с малым размером кода, это облегчает использование библиотеки не очень искушенными пользователями.

Однако эти достоинства порождают недостатки библиотеки, такие как меньшее число виджетов, несколько упрощенная графика и невозможность сборки приложения, выглядящего естественно под конкретной операционной системой.
\subsubsection{Название}
Изначально назывался FL (Forms Library). При переходе в open source выяснилось, что поиск по названию FL практически невозможен — аббревиатура FL также означает штат Флорида. Поэтому пакет был переименован в FLTK (FL Toolkit), позднее ему был придуман бэкроним Fast, Light Toolkit.
\subsubsection{История}
FLTK начал разрабатываться как замена библиотеке XForms, а позднее был портирован на Mac OS и Windows. FLTK появился раньше, чем другие популярные библиотеки для создания GUI, но был практически неизвестен до 1998 года.
\subsubsection{Особенности}
FLTK представляет собой библиотеку виджетов и работает на ОС UNIX/Linux X11, Microsoft Windows и MacOS X. Малый объём библиотеки делает её подходящей для использования во встраиваемых системах.

Для встраиваемых систем на основе embedded Linux возможны следующие варианты:

FLTK + nxlib + nano-X (довольно стабильно работает, но есть проблемы с кириллицей)

FLNX — порт FLTK 1.0.7 на nano-X (работает только с версией 0.92)

DirectFB FLTK — порт FLTK на DirectFB + собственно сам DirectFB (данная сборка нестабильная, шрифты необходимо инсталлировать как для X11 и указать путь в конфиге)

\section{Visual Studio Code}
\textbf{Visual Studio Code} (VS Code)~\cite{wikiVS} — текстовый редактор, разработанный Microsoft для Windows, Linux и macOS. Позиционируется как «лёгкий» редактор кода для кроссплатформенной разработки веб- и облачных приложений. Включает в себя отладчик, инструменты для работы с Git, подсветку синтаксиса, IntelliSense и средства для рефакторинга. Имеет широкие возможности для кастомизации: пользовательские темы, сочетания клавиш и файлы конфигурации. Распространяется бесплатно, разрабатывается как программное обеспечение с открытым исходным кодом, но готовые сборки распространяются под проприетарной лицензией.

Visual Studio Code основан на Electron и реализуется через веб-редактор Monaco, разработанный для Visual Studio Online.
\subsubsection{История}
Visual Studio Code был анонсирован 29 апреля 2015 года компанией Microsoft на конференции Build, и вскоре была выпущена бета-версия.

18 ноября 2015 года Visual Studio Code был выпущен под лицензией MIT, а исходный код был опубликован на GitHub. Анонсирована поддержка расширений.

14 апреля 2016 года Visual Studio Code вышел из стадии бета-тестирования.
\subsubsection{Возможности}
Visual Studio Code — это редактор исходного кода. Он имеет многоязычный интерфейс пользователя и поддерживает ряд языков программирования, подсветку синтаксиса, IntelliSense, рефакторинг, отладку, навигацию по коду, поддержку Git и другие возможности. Многие возможности Visual Studio Code недоступны через графический интерфейс, зачастую они используются через палитру команд или JSON-файлы (например, пользовательские настройки). Палитра команд представляет собой подобие командной строки, которая вызывается сочетанием клавиш.

VS Code также позволяет заменять кодовую страницу при сохранении документа, символы перевода строки и язык программирования текущего документа.

С 2018 года появилось расширение Python для Visual Studio Code с открытым исходным кодом. Оно предоставляет разработчикам широкие возможности для редактирования, отладки и тестирования кода.

Также VS Code поддерживает редактирование и выполнение файлов типа «Блокнот Jupyter» (Jupyter Notebook) напрямую «из коробки» без установки внешнего модуля в режиме визуального редактирования и в режиме редактирования исходного кода.

На март 2019 года посредством встроенного в продукт пользовательского интерфейса можно загрузить и установить несколько тысяч расширений только в категории «programming languages» (языки программирования).

Также расширения позволяют получить более удобный доступ к программам, таким как Docker, Git и другие. В расширениях можно найти линтеры кода, темы для редактора и поддержку синтаксиса отдельных языков.

Visual Studio Code имеет поддержку плагинов, доступных через Visual Studio Marketplace. Они могут включать в себя дополнения к редактору, поддержку дополнительных языков программирования, статические анализаторы кода.

С мая 2019 года доступен закрытый тест редактора Visual Studio Online на основе VS Code. Он поддерживает все расширения и IntelliCode.
\section{Python}
Python~\cite{wikiPy} (в русском языке встречаются названия пито́н или па́йтон) — высокоуровневый язык программирования общего назначения с динамической строгой типизацией и автоматическим управлением памятью, ориентированный на повышение производительности разработчика, читаемости кода и его качества, а также на обеспечение переносимости написанных на нём программ. Язык является полностью объектно-ориентированным в том плане, что всё является объектами. Необычной особенностью языка является выделение блоков кода отступами. Синтаксис ядра языка минималистичен, за счёт чего на практике редко возникает необходимость обращаться к документации. Сам же язык известен как интерпретируемый и используется в том числе для написания скриптов. Недостатками языка являются зачастую более низкая скорость работы и более высокое потребление памяти написанных на нём программ по сравнению с аналогичным кодом, написанным на компилируемых языках, таких как C или C++.

Python является мультипарадигменным языком программирования, поддерживающим императивное, процедурное, структурное, объектно-ориентированное программирование, метапрограммирование, функциональное программирование и асинхронное программирование. Задачи обобщённого программирования решаются за счёт динамической типизации. Аспектно-ориентированное программирование частично поддерживается через декораторы, более полноценная поддержка обеспечивается дополнительными фреймворками. Такие методики как контрактное и логическое программирование можно реализовать с помощью библиотек или расширений. Основные архитектурные черты — динамическая типизация, автоматическое управление памятью, полная интроспекция, механизм обработки исключений, поддержка многопоточных вычислений с глобальной блокировкой интерпретатора (GIL), высокоуровневые структуры данных. Поддерживается разбиение программ на модули, которые, в свою очередь, могут объединяться в пакеты.

Эталонной реализацией Python является интерпретатор CPython, который поддерживает большинство активно используемых платформ, являющийся стандартом де-факто языка. Он распространяется под свободной лицензией Python Software Foundation License, позволяющей использовать его без ограничений в любых приложениях, включая проприетарные. CPython компилирует исходные тексты в высокоуровневый байт-код, который исполняется в стековой виртуальной машине. К другим трём основным реализациям языка относятся Jython (для JVM), IronPython (для CLR/.NET) и PyPy. PyPy написан на подмножестве языка Python (RPython) и разрабатывался как альтернатива CPython с целью повышения скорости исполнения программ, в том числе за счёт использования JIT-компиляции. Поддержка версии Python 2 закончилась в 2020 году. На текущий момент активно развивается версия языка Python 3. Разработка языка ведётся через предложения по расширению языка PEP (англ. Python Enhancement Proposal), в которых описываются нововведения, делаются корректировки согласно обратной связи от сообщества и документируются итоговые решения.

Стандартная библиотека включает большой набор полезных переносимых функций, начиная с возможностей для работы с текстом и заканчивая средствами для написания сетевых приложений. Дополнительные возможности, такие как математическое моделирование, работа с оборудованием, написание веб-приложений или разработка игр, могут реализовываться посредством обширного количества сторонних библиотек, а также интеграцией библиотек, написанных на Си или C++, при этом и сам интерпретатор Python может интегрироваться в проекты, написанные на этих языках[14]. Существует и специализированный репозиторий программного обеспечения, написанного на Python, — PyPI. Данный репозиторий предоставляет средства для простой установки пакетов в операционную систему и стал стандартом де-факто для Python. По состоянию на 2019 год в нём содержалось более 175 тысяч пакетов.

Python стал одним из самых популярных языков, он используется в анализе данных, машинном обучении, DevOps и веб-разработке, а также в других сферах, включая разработку игр. За счёт читабельности, простого синтаксиса и отсутствия необходимости в компиляции язык хорошо подходит для обучения программированию, позволяя концентрироваться на изучении алгоритмов, концептов и парадигм. Отладка же и экспериментирование в значительной степени облегчаются тем фактом, что язык является интерпретируемым. Применяется язык многими крупными компаниями, такими как Google или Facebook.

